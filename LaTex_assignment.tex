\documentclass{article}
\usepackage{graphicx}
\title{Bisection method and Newton method to find solution to non-linear equations}
\author{\textbf{\textit{Ashraf Mohammed Hassan Anil  SCT211-0255/2021}}}
\date{March, 18, 2023}

\begin{document}
\maketitle
\begin{Large}\textbf{ICS2207: Scientific Computing}
\end{Large}\\[12pt]

The non-linear equation used by both methods to find its roots was:
\textbf{$$f(x) = x^6 + x - 1$$}


\textbf{timeit()} in python was used to measure time taken by both methods

\section{Bisection Method for Finding Roots of a Function}

\subsection{Implementation of Bisection method in Python}

Here's an example Python code for implementing the Bisection method:


\begin{verbatim}
# import math
import timeit


def f(x):
    return x ** 6 + x - 1


def bisection_method():
    a, b = 0, 2
    tolerance = 1e-6
    max_iteration = 1000
    count_iteration = 0
    if f(a) * f(b) >= 0:
        print("Root cannot be found; range[a, b] should generate f(a) and f(b) with opposite signs")
    for i in range(max_iteration):
        count_iteration += 1
        c = (a + b) / 2
        if abs(f(c)) < tolerance:
            print(f"Number of iterations taken are {count_iteration}")
            print(f"Root found at x={c:.6f}")
            break

        elif f(c) * f(a) < 0:
            b = c
        else:
            a = c


bisection_method()

time_taken = timeit.timeit(bisection_method, number=10000)

print(f"Bisection method took an average of {time_taken / 10000:.6f} seconds per run over 10000 runs.")

\end{verbatim}

\subsection{Explanation of the code}

The interval $ a = 0$ and $b = 2$ was chosen since $f(a) = 0$ while $f(b) = 65$ thus $f(a)$ and $f(b)$ have opposite signs thus it holds\\[10pt]
The program returns the number of iterations required to reach the root and the average time the bisection method takes in 1000 runs

\section{Newton-Raphson Method for Finding Roots of a Function}

\subsection{Implementation of Newton-Raphson method in Python}

Here's an example Python code for implementing the Newton-Raphson method:

$f(x) = x^6 + x - 1$

\begin{verbatim}
import math
import timeit


def f(x):
    return x**6 + x - 1


def df(x):
    return 6*x**5 + 1


def newton():
    x0 = 3
    tolerance = 1e-6
    maxiterr = 1000
    count_iterations = 0

    for i in range(maxiterr):
        count_iterations += 1
        fx = f(x0)
        dfx = df(x0)

        x1 = x0 - (fx / dfx)

        if abs(f(x1)) < tolerance:
            print(f"Number of iterations taken = {count_iterations}")

            print(f"Root found at x = {x1:.6f}")

            break
        else:
            x0 = x1


time_taken = timeit.timeit(newton, number=10000)

print(f"Newton method took an average of {time_taken/10000:.6f} seconds per run over 10000 runs.")

\end{verbatim}

\subsection{Explanation of the code}

From the above code $x_n$ was taken to be 2 since from the graph of $f(x)$ plotted against $x$ the initial guess is to be obtained from the range of about -7.5 to 7.5 so that the method generates the roots.\\[11pt]
However, the closer to zero the value is obtained from the better hence 2 was taken and not values like 4, 5, 6 etc.\\[11pt]
The program returns the number of iterations required to reach the root and the average time the bisection method takes in 1000 runs


\textbf{Figure below is the plot of $f(x)$ against $x$ }
\begin{figure}[ht]
\centering
\includegraphics[width=0.8\textwidth]{newton_fig.png}
\caption{Plot of $f(x) = x^6 + x - 1$}
\label{fig:myplot}
\end{figure}


\subsection{Comparison between Bisection method and Newton method}
The Newton-Raphson method converges faster than the bisection method. This is because the Newton-Raphson method used less number of iterations: 4 under 0.000021s while Bisection took: 19 iteration in 0.000030s\\[12pt]
However, the Newton-Raphson method can be sensitive to the initial guess, and can converge to a different root or fail to converge altogether if the initial guess is not close enough to the true root. On the other hand, the bisection method always converges to a root, but it does so more slowly and may require more iterations to achieve the same level of accuracy.

\subsection{Conclusion}
Root obtained by both methods was 0.778090 which when substituted in 
$f(x) = x^6 + x - 1$ results to $f(x) = 0.00000108806$ which is a valid solution as it is close to 0 within the given tolerance


\end{document}
